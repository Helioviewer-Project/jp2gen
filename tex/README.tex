%&latex
%&latex
\documentclass[namedreferences]{SolarPhysics}
\usepackage[optionalrh]{spr-sola-addons} % For Solar Physics 
%\usepackage{epsfig}          % For eps figures, old commands
\usepackage{graphicx}        % For eps figures, newer & more powerfull
%\usepackage{courier}         % Change the \texttt command to courier style
%\usepackage{natbib}         % For citations: redefine \cite commands
%\usepackage{amssymb}        % useful mathematical symbols
\usepackage{color}           % For color text: \color command
\usepackage{url}             % For breaking URLs easily trough lines
\def\UrlFont{\sf}            % define the fonts for the URLs


% General definitions
% please place your own definitions here and don't use \def but
% \newcommand{}{} or 
% \renewcommand{}{} if it is already defined in LaTeX

\newcommand{\BibTeX}{\textsc{Bib}\TeX}
\newcommand{\etal}{{\it et al.}}

% Definitions for equations
\newcommand{\deriv}[2]{\frac{{\mathrm d} #1}{{\mathrm d} #2}}
\newcommand{\rmd}{ {\ \mathrm d} }
\renewcommand{\vec}[1]{ {\mathbf #1} }
\newcommand{\uvec}[1]{ \hat{\mathbf #1} }
\newcommand{\pder}[2]{ \f{\partial #1}{\partial #2} }
\newcommand{\grad}{ {\bf \nabla } }
\newcommand{\curl}{ {\bf \nabla} \times}
\newcommand{\vol}{ {\mathcal V} }
\newcommand{\bndry}{ {\mathcal S} }
\newcommand{\dv}{~{\mathrm d}^3 x}
\newcommand{\da}{~{\mathrm d}^2 x}
\newcommand{\dl}{~{\mathrm d} l}
\newcommand{\dt}{~{\mathrm d}t}
\newcommand{\intv}{\int_{\vol}^{}}
\newcommand{\inta}{\int_{\bndry}^{}}
\newcommand{\avec}{ \vec A}
\newcommand{\ap}{ \vec A_p}
\newcommand{\bb}{ \vec B}
\newcommand{\jj}{ \vec j}
\newcommand{\rr}{ \vec r}
\newcommand{\xx}{ \vec x}

% Definitions for the journal names
\newcommand{\adv}{    {\it Adv. Space Res.}}
\newcommand{\annG}{   {\it Annales Geophysicae}}
\newcommand{\aap}{    {\it Astron. Astrophys.}}
\newcommand{\aaps}{   {\it Astron. Astrophys. Suppl.}}
\newcommand{\aapr}{   {\it Astron. Astrophys. Rev.}}
\newcommand{\ag}{     {\it Ann. Geophys.}}
\newcommand{\aj}{     {\it Astron. J.}}
\newcommand{\apj}{    {\it Astrophys. J.}}
\newcommand{\apss}{   {\it Astrophys. Space Sci.}}
\newcommand{\cjaa}{   {\it Chin. J. Astron. Astrophys.}}
\newcommand{\gafd}{   {\it Geophys. Astrophys. Fluid Dyn.}}
\newcommand{\grl}{    {\it Geophys. Res. Lett.}}
\newcommand{\ijga}{   {\it Int. J. Geomag. Aeron.}}
\newcommand{\jastp}{  {\it J. Atmos. Solar Terr. Phys.}}
\newcommand{\jgr}{    {\it J. Geophys. Res.}}
\newcommand{\mnras}{  {\it Mon. Not. Roy. Astron. Soc.}}
\newcommand{\nat}{    {\it Nature}}
\newcommand{\pasp}{   {\it Pub. Astron. Soc. Pac.}}
\newcommand{\pasj}{   {\it Pub. Astron. Soc. Japan}}
\newcommand{\pre}{    {\it Phys. Rev. E}}
\newcommand{\solphys}{{\it Solar Phys.}}
\newcommand{\sovast}{ {\it Sov. Astron.}}
\newcommand{\ssr}{    {\it Space Sci. Rev.}}

\newcommand{\cvar}[1]{$<${\it #1}$>$}
\newcommand{\sname}{HVF2J}

%%%%%%%%%%%%%%%%%%%%%%%%%%%%%%%%%%%%%%%%%%%%%%%%%%%%%%%%%%%%%%%%%%
\begin{document}

\begin{article}

\begin{opening}

\title{The Helioviewer Suite of FITS to JP2 Programs - version 20090501}

\author{J.~\surname{Ireland}$^{1}$\sep
        D.~M.~\surname{M\"uller}$^{1}$\sep
        V.~K.~\surname{Hughitt}$^{1}$      
       }
\runningauthor{J. Ireland et al.}
\runningtitle{Helioviewer: FITS 2 JP2 - \sname}

   \institute{$^{1}$ ADNET Systems, Inc., at NASA's GSFC
                     email: \url{webmaster} email: \url{webmaster}\\ 
              $^{2}$ ESA RSSD at NASA's GSFC
                     email: \url{webmaster} \\
             }

\begin{abstract}
If you have the FITS files, then you also can get the JP2 files.  Read
on to find out more!
\end{abstract}
\keywords{?}
\end{opening}
%-------------------------------------------------

\section{Introduction}
The software provided here creates the JPEG2000 images (extension .jp2)
and storage structure required by the Helioviewer project, from solar
observational data.  We will denote the whole software package by
\sname\ - Helioviewer FITS to JPEG2000.


\section{Required software NOT provided by \sname}\label{sec:req}

You will need Solarsoft, which can be obtained at
\url{http://www.lmsal.com/ solarsoft}.  Please note that if you want
to produce JP2 images for a particular instrument, then you should
make sure your Solarsoft installation contains the necessary code and
databases in order for you make such an image.  This may mean that you
need the full calibration software in order to turn a source FITS file
into a calibrated FITS file.  Or it may be something as simple as
being able to rotate and crop the image as required (such as MDI).
please consult the relevant instrument team for more detail on how to
produce well calibrated, useful images.

If you require images some portion of which is transparent, then we
recommend using the Kakadu JPEG2000 library.  Note that it should not
be necessary to have access to the source code - the binary for your
JP2 production platform should be sufficient.



\section{Structure of the suite}\label{sec:structure}

The suite has the following structure

\begin{itemize}
\item progs - main directory
\item progs/gen - programs used by all the instrument processing programs
\item progs/gen/jp2 - general programs for JP2 creation
\end{itemize}


There are also a number of subdirectories which refer to a particular
instrument:
\begin{itemize}
\item progs/\cvar{instrument} -  programs specific to \cvar{instrument} 
\item progs/\cvar{instrument}/jp2 - FITS to JP2 programs for \cvar{instrument} 
\item progs/\cvar{instrument}/gen - programs used by all \cvar{instrument}  detectors
\end{itemize}

Running the routines with defaults will create a directory
\begin{itemize}
\item jp2/ 
\end{itemize}
which the main directory where all the JP2 files will be stored.  The
JP2 files are ordered by time and then by observatory > instrument >
detector > measurement.



\section{Currently supported observers and measurements}

Each image is produced by some piece of equipment which can be
categorised as
\begin{verbatim}
observatory > instrument > detector > measurement
\end{verbatim}
where $>$ roughly means ``is a superset of'', or ``contains'', and is
meant to describe the structure of a typical {\it observer} and its
relation to its {\it measurements}.  An {\it observer} is constructed
as
\begin{verbatim}
observer = observatory + instrument + detector
\end{verbatim}
For example, SOH-LAS-0C2 is a different observer from SOH-LAS-0C3,
which is different again from SOH-EIT-EIT.  Each observer can make
multiple different types of {\it measurement}.

\begin{table}\label{tab:supported}
\begin{tabular}{ccccc}
observatory & instrument & detector & measurement           & supported? \\ \hline
SOHO          & EIT              & EIT         & 304, 171, 195, 284   &   yes      \\
                   & MDI            & MDI        & int, mag                    &   yes      \\
                   & LASCO        & C2          & WL                            &   yes      \\
                   &                   & C3           & WL                            &   yes      \\
                   & CDS            & NIS         & many wavlengths      & desired \\ 
                   & SUMER        & SUMER    & many wavlengths      & desired \\  \hline
TRACE         & TRACE        & TRACE    & 171, 195                   & in development  \\\hline
Hinode        & XRT            & XRT         & sxr                           & desired   
\end{tabular}
\caption{Supported observatories, instruments, detectors and
  measurements.  If you have a stable contribution that covers
  something we don't, please let us know.}
\end{table}


\section{Setting up your distribution of \sname}



\subsection{Storing your JP2 files - JI\_HV\_STORAGE}

This file is located in progs/gen/jp2.  It is a procedure which
returns a structure with the follwing two tags:
\begin{itemize}
\item \cvar{jp2\_location} - root directory for the JPEG2000 files.  The
  default location is relative = '../../../jp2/'.  Note that the
  trailing slash is required.  The default assumes that you are
  running from an \cvar{observer} subdirectory in progs, e.g.,
  progs/eit/jp2/ .

\item \cvar{hvs\_location} - root directory for log files. The log
  files list which FITS files have been converted to JPEG200
  files. The default location is relative = '../../../hvs/'.  Note
  that the trailing slash is required.  The default assumes that you
  are running from an \cvar{observer} subdirectory in progs, e.g.,
  progs/eit/jp2/ .
\end{itemize}

\subsection{Letting everyone know who wrote your JP2 files - JI\_HV\_WRITTENBY}

This file is located in progs/gen/jp2.  It is a procedure which
returns a structure with the follwing three tags:
\begin{itemize}
\item \cvar{institute} - the name of your home institute

\item \cvar{contact} - an email address to allow users of your
  JPEG2000 to contact you

\item \cvar{kdu\_lib\_location} - where the kakdu library is installed
  at your home institution.

\end{itemize}
Every JPEG2000 file produced contains a \cvar{institute} and
\cvar{contact} tag.  In a distributed architecture, it is useful to
know where a particular file was written and whom someone should
contact in case of problems.  The variable \cvar{kdu\_lib\_location}
is also included to allow users to put the KDU code wherever they
like.  Note also that on OS X you may also have to change the
environment variable concerning the location of dynamic libraries.

\subsection{JP2 compression parameters as a function of
  \cvar{observer} and \cvar{measurement}- JI\_HV\_OBSERVER\_DETAILS}

This file returns parameter choices for use with the encoding of JP2
files.  It returns suitable parameter choices for each
\cvar{observer}-\cvar{measurement} pair.  Currently supported
\cvar{observer}-\cvar{measurement} partners are given in Table
\ref{tab:supported}. 

Default choices for encoding of the JP2 files are as follows.

\begin{itemize}
\item {\it n\_layers = 8}  - the name of your home institute

\item  {\it n\_levels = 8 } - an email address to allow users of your
  JPEG2000 to contact you

\item  {\it bit\_rate = [0.5,0.01] } - where the kakdu library is installed
  at your home institution.

\end{itemize}
For more details as to the meaning of these parameters, please consult
\url{http://www.helioviewer.org/wiki}. 

Note that we will endeavor to keep JI\_HV\_OBSERVER\_DETAILS as
up-to-date as possible.  If you have any changes to suggest, or code
supporting different \cvar{observer}-\cvar{measurement} partners, then
please let us know, and we will include your code in the main distribution.


\subsection{Generating EIT JP2 images from your EIT FITS files}




\subsection{Generating MDI JP2 images from your MDI FITS files}

\subsection{Generating LASCO -C2 and -C3 JP2 images from your LASCO FITS files}



\section{General principles for writing your own programs to turn your
  FITS files into JP2 files for use with the Helioviewer Project}


\end{article}

\end{document}
